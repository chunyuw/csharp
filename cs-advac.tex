%% Slides for ".NET Programming" by Chunyu Wang <chunyu@hit.edu.cn> %%

\part{C\# 高级特性}
\begin{frame}
\frametitle{Outline}            % {C\# 高级特性}
\tableofcontents
\end{frame}

\section{抽象类}

\begin{frame}[fragile]
\frametitle{抽象类}
\begin{block}{抽象类}
  \CJKindent 未完全实现的类,其中的方法和特性等可以只声明不实现。只能用于其他类
  的基类,其派生类可以是抽象类,否则必须实现继承来的抽象方法。
\end{block}
\pause
\begin{itemize}
\item 抽象类不能生成实例,需要\texttt{abstract} 修饰
\item 抽象方法只给出方法的声明,无具体实现,也需要\texttt{abstract}
\item 而具体实现由派生类完成
\end{itemize}
\lstset{emph={abstract}}
\begin{lstlisting}
public abstract class Graphics{
  public static double pi = 3.14159;
  public abstract double Area();
  public abstract double Perimeter();
}
\end{lstlisting}
\end{frame}

\begin{frame}[fragile]
\frametitle{抽象方法}
\lstset{emph={abstract}}
\begin{lstlisting}
public abstract class Graphics{
  public static double pi = 3.14159;
  public abstract double Area();
  public abstract double Perimeter();
}
\end{lstlisting}
\lstset{emph={override}}
\begin{lstlisting}
class Circle:Graphics {
  public int r = 2;
  public override double Perimeter(){
    return 2 * pi * r;
  }
  public override double Area(){
    return pi * r * r;
  }
}
\end{lstlisting}
\end{frame}

\begin{frame}[fragile]
\frametitle{抽象特性}
\lstset{emph={abstract}}
\begin{lstlisting}
public abstract class Shape {
  private string m_id;
  public Shape(string s)    { Id = s; }
  public string Id
    { get { return m_id; }
      set { m_id = value; } }

  public abstract double Area
    { get; }
}
\end{lstlisting}
\lstset{emph={override}}
\begin{lstlisting}
public class Square : Shape {
...
 public override double Area
 { get { return m_side * m_side; } }
}
\end{lstlisting}
\end{frame}

\begin{frame}[fragile]
\frametitle{密封类}
使用 \texttt{sealed} 修饰的类,不可以被继承:
\lstset{emph={sealed}}
\begin{lstlisting}
public sealed class A {
  // ...
}

public class B : A { // ERROR!
  // ...
}
\end{lstlisting}

\end{frame}

\begin{frame}[fragile]
\frametitle{静态类}
使用 \texttt{static} 修饰的类,只有静态成员,不能被实例化:
%\lstset{emph={static}}
\begin{lstlisting}[escapeinside=<>]
public static class NumericFn {
  public static double PI = 3.1415;
  public static double reciprocal(double num) {
    return 1/num;
  }
  static public double fracPart(double num) {
    return num - (int) num;
  }
  static public bool isEven(double num) {
    return (num % 2) == 0 ? true : false;
  }
  static public bool isOdd(double num) { ... }
  ...
}
\end{lstlisting}
\end{frame}

\begin{frame}[fragile]
\frametitle{内嵌类}
\begin{itemize}
\item 在类内部定义的类
\begin{lstlisting}[escapeinside=<>]
public class OuterClass     //<称为外部类>
{  ...
  public class NestedClass  //<称为内嵌类>
  {  // Insert code to implement NestedClass.
  }
  public void func()
  { ... NestedClass x = new NestedClass(); ... }
}
...
  OuterClass.NestedClass y = new OuterClass.NestedClass();
\end{lstlisting}
\item 在外部类中直接使用内嵌类
\item 在外部类外面访问类,使用全名
  \begin{itemize}
  \item \texttt{OuterClass.NestedClass}
  \end{itemize}
\item 访问权限由\texttt{OuterClass} 和\texttt{NestedClass} 共同决定
\end{itemize}
\end{frame}

\section{接口}

\begin{frame}[fragile]
\frametitle{接口}
  \begin{block}{\textit{interface}}
    \CJKindent 能声明特性、索引器、事件和方法的特殊结构,只为这些成员提供声明,
    没有具体实现。
  \end{block}
接口的一般定义
\begin{lstlisting}
[attributes]
[modifiers] interface identifier [:baselist]
{ interface_body } [;]
\end{lstlisting}
\begin{itemize}
\item 接口声明的语法和类基本相同
\item 接口实际也是一种类型
\item 接口的实现由类完成
\end{itemize}
\end{frame}

\begin{frame}[fragile,t]
\frametitle{接口示例}

\begin{onlyenv}<1-2| handout:1>
  \lstset{emph={interface}}
\begin{lstlisting}
public interface IMeasurement{
  double Area();
  double Perimeter();
}
\end{lstlisting}
\end{onlyenv}
\begin{onlyenv}<1| handout:1>
  \lstset{emph={[2]Perimeter,Area}}
\begin{lstlisting}
class    Circle:IMeasurement {
  public int r = 2;
  public static double pi = 3.14159;
  public double Perimeter(){
    return 2 * pi * r;
  }
  public double Area(){
    return pi * r * r;
  }
}
\end{lstlisting}
\end{onlyenv}
\begin{onlyenv}<2| handout:0>
  \lstset{emph={[2]Perimeter,Area}}
\begin{lstlisting}
class Rectangle:IMeasurement {
  public int width,height;
  public double Perimeter(){
    return 2 * (width + height);
  }
  public double Area(){
    return width * height;
  }
}
\end{lstlisting}
\end{onlyenv}
\end{frame}

\begin{frame}[fragile]
\frametitle{接口的特点}
\begin{itemize}
%\setlength{\itemsep}{8pt plus 1pt}
\item 为类提供了蓝图,规范了类之间的交互方式
\item 通过继承其他接口,可是实现接口的重用
\item 和抽象类一样,不可以生成实例
\item 接口成员默认为 public 和 abstract
\item 接口也是一种类型,可以定义变量,访问子类对象
\begin{lstlisting}[escapeinside=<>]
IMeasurement x = new Recangle(); 
                          // <使用了隐式转换>
x.Area();                 // x<可以直接访问>Area()<方法>
\end{lstlisting}
\item 也可以使用 is 和 as 操作符判断接口类型
\begin{lstlisting}
if (o is IMeasurement) ...
IMeasurement x = o as IMeasurement;
\end{lstlisting}
\end{itemize}
\end{frame}

\begin{frame}[fragile]
\frametitle{接口和抽象类的区别}
\begin{itemize}
\setlength{\itemsep}{8pt plus 1pt}
\item 只有四种成员,仅提供定义而不实现
  \begin{itemize}
  \item 特性、索引器、事件和方法
  \end{itemize}
\item 定义不需要 public 和 abstract,实现不需要 override
\item 类只能继承一个类,但可以实现多个接口

  \lstset{emph={[2]First,Second}}
\begin{lstlisting}
public interface First   {...}
public interface Second {...}
class MyClass : First, Second
  { ... }
\end{lstlisting}
\end{itemize}
\end{frame}

% \begin{frame}
% \frametitle{接口的作用}
% \begin{itemize}
% \setlength{\itemsep}{8pt plus 1pt}
% \item 接口规定了类的一种行为方式,规范了类的行为
% \item 实现某一接口的各种类,在这方面的行为是一致的
% \item 通过接口连接的模块,代码之间的耦合度小
% \item 系统提供了大量的现成接口
% \begin{itemize}
% \setlength{\itemsep}{6pt plus 1pt}
% \item System.IComparable --- 其中的 CompareTo() 用于比较
% \item System.Collections.ICollection --- 集合操作
% \item System.Collections.IEnumerator --- 迭代操作
% \end{itemize}
% \end{itemize}
% \end{frame}

\begin{frame}[fragile]
\frametitle{接口中的特性}
实现一个自动返回序列的特性:
\begin{lstlisting}
using System;
public interface ISeries {  // An interface property.
  int next { get; // return the next number in series
             set; // set next number
  }
}
class ByTwos : ISeries {
  int val;
  public ByTwos() { val = 0; }

  public int next {
    get { val += 2;
          return val;  }
    set {
          val = value; }
  }
}

\end{lstlisting}

\end{frame}

\begin{frame}[fragile]
\frametitle{接口中的索引器}
\begin{lstlisting}
using System;

public interface ISeries {  // an interface property
  ...
  // an interface indexer
  int this[int index] {
    get; // return the specified number in series
  }
}
...
  // get a value using an index
  public int this[int index] {
    get {
      val = 0;
      for(int i=0; i<index; i++)
        val += 2;
      return val;
    }
  }

\end{lstlisting}

\end{frame}

\begin{frame}[fragile]
\frametitle{接口示例}

\begin{itemize}
\item 线性表类LinkedList 和接口Linkable (程序LinkedList.cs)
%\attachfile[description=Interface example]{code/LinkedList.cs}

\begin{lstlisting}[escapeinside=<>]
interface Linkable {
  Linkable Next { get; set; }  
}

class LinkedList  { // <使用> Linkable <接口>
  Linkable head;
  public Linkable Head { get { return head; } }

  public void insertAtHead(Linkable node) { ... }
  public void insertAtTail(Linkable node) { ... }
  public Linkable removeFromHead()        { ... }
  public Linkable removeFromTail()        { ... }
}

class XXX: Linkable { ... } // <实现> Linkable <接口>
\end{lstlisting}
\end{itemize}

\end{frame}

\begin{frame}[fragile]
\frametitle{接口的继承}
\begin{lstlisting}
public interface A {
  void meth1();
  void meth2();
}

// B now includes meth1() and meth2() -- it adds meth3().
public interface B : A {
  void meth3();
}

\end{lstlisting}
\begin{itemize}
\item 接口中也可以使用 new 关键字隐藏基接口成员
\end{itemize}
\end{frame}

\begin{frame}[fragile]
\frametitle{显式实现接口}
 接口默认是 public,如果需要 private 要使用显式实现
\begin{lstlisting}
using System;
interface IEven {
  bool isOdd(int x);
  bool isEven(int x);
}
class MyClass : IEven {
  // Explicit implementation. Notice that this
  // member is private by default.
  bool IEven.isOdd(int x) {
    if((x%2) != 0) return true;
    else return false;
  }
...
}
\end{lstlisting}

\end{frame}

\begin{frame}[fragile]
\frametitle{显式实现消除歧义}
\begin{lstlisting}
using System;
interface IMyIF_A {
  int meth(int x);
}
interface IMyIF_B {
  int meth(int x);
}
class MyClass : IMyIF_A, IMyIF_B {

  // explicitly implement the two meth()s
  int IMyIF_A.meth(int x) {
    return x + x;
  }
  int IMyIF_B.meth(int x) {
    return x * x;
  }
...
}
\end{lstlisting}

% \begin{frame}[fragile]
% \frametitle{迭代器}
% \begin{itemize}
% \item System.Collections.IEnumerable 接口 (程序IEnumerable.cs)
% \begin{lstlisting}[escapeinside=<>]
% interface IEnumerable 
% {
%   IEnumerator GetEnumerator();
% }
% interface IEnumerator
% {
%   object Current { get; }
%   bool MoveNext();
%   void Reset();
% }
% \end{lstlisting}
% \item 常用于foreach 循环的迭代
% \end{itemize}
% \end{frame}

\end{frame}

\section{委托}

\begin{frame}[fragile]
\frametitle{委托}
  \begin{block}{\textit{delegate}}
    \CJKindent 具有特定返回类型和参数列表的方法的引用类型;方法名字任意,可以是
    静态或实例方法,只要类型符合;对方法实际能进行的操作没有限制。
  \end{block}
委托定义的一般格式:
\begin{lstlisting}
[modifiers] delegate <returntype> <name> ([parameter list]);
\end{lstlisting} \pause

  \begin{itemize}
  \item 委托也是一种类型,用于包装方法
  \item 定义委托后,需要声明,并创建委托的实例
  \item 委托的实例,相当于方法的容器,可以添加或去掉方法
  \item 调用委托,相当于调用其中的全部方法
  \item 通过委托,程序可以在运行时动态的调用不同的方法
  \end{itemize}
\end{frame}

\begin{frame}[fragile]
\frametitle{委托示例(1)}
\begin{lstlisting}[escapeinside=<>]
// <定义委托 Compute>
delegate int Compute (int a, int b);

class Test {
  public Compute comp;        // <声明委托>
  public int add (int x, int y)
    { return x + y; }
}
...
  Test n = new Test();
  n.comp = new Compute(n.add); // <创建委托实例>

  int r  = n.comp (20,40);    // <调用委托>

  System.Console.WriteLine (r);
\end{lstlisting}

\end{frame}

\begin{frame}[fragile]
\frametitle{委托示例(2)}
\lstset{emph={[2]replaceSpaces,removeSpaces,reverse}}
\begin{lstlisting}
using System;
delegate string StrMod(string str);
class DelegateTest {
  static string replaceSpaces(string a) {
    Console.WriteLine("Replaces spaces with hyphens.");
    return a.Replace(' ', '-');   }
  static string removeSpaces(string a) {
    string temp = "";  int i;
    Console.WriteLine("Removing spaces.");
    for(i=0; i < a.Length; i++)
      if(a[i] != ' ') temp += a[i];
    return temp; }
  static string reverse(string a) {
    string temp = ""; int i, j;
    Console.WriteLine("Reversing string.");
    for(j=0, i=a.Length-1; i >= 0; i--, j++)
      temp += a[i];
    return temp;
}
\end{lstlisting}
\end{frame}
\begin{frame}[fragile]
\frametitle{委托示例(2续)}
\begin{lstlisting}[escapeinside=<>]
public static void Main() {
  // <创建委托>
  StrMod strOp = new StrMod(replaceSpaces);
  string str;

  // <调用委托>
  str = strOp("This is a test.");
  Console.WriteLine("Resulting string: " + str);
  Console.WriteLine();

  strOp = new StrMod(removeSpaces);
  str = strOp("This is a test.");
  Console.WriteLine("Resulting string: " + str);
  Console.WriteLine();

  strOp = new StrMod(reverse);
  str = strOp("This is a test.");
  Console.WriteLine("Resulting string: " + str);
}

\end{lstlisting}
\end{frame}

\begin{frame}
\frametitle{委托的优势}

\begin{itemize}
\setlength{\itemsep}{6pt plus 1pt}
\item 传统的 Microsoft Windows 使用函数指针进行回调
\item 函数指针指向的地址不一定有效,容易发生错误
\item C\# 则通过类型安全的委托进行回调
\item 编译器完成类型检查,确保其中的方法可靠
\item 不需使用函数指针,易于维护
\end{itemize}

\end{frame}

\begin{frame}[fragile]
\frametitle{自动类型转换}
\CJKindent C\# 2.0 中,可以省略 new 表达式,直接使用方法名创建委托实例,添加委托
\begin{lstlisting}[escapeinside=<>]
public static void Main() {
  // <自动将方法转换为委托>
  StrMod strOp = replaceSpaces;
  string str;

  // <委托调用>
  str = strOp("This is a test.");

  strOp = removeSpaces; // <自动转换>
  str = strOp("This is a test.");

  strOp = reverse; // <自动转换>
  str = strOp("This is a test.");
}
\end{lstlisting}
\end{frame}


\begin{frame}[fragile]
\frametitle{多播委托}
\begin{lstlisting}
public static void Main() {

  StrMod strOp;
  string str = "This is a test";

  // Set up multicast.
  strOp = replaceSpaces;
  strOp += reverse;

  // Call multicast.
  strOp(str);

  strOp -= replaceSpaces;
  strOp += remove;

  strOp(str);
}
\end{lstlisting}
\end{frame}

\begin{frame}[fragile]
\frametitle{多播委托}
\begin{itemize}
\setlength{\itemsep}{4pt plus 1pt}
\item<1-| handout:1> 多播委托中,委托相当于一个方法的调用列表
\item<2-| handout:1> 通过 \texttt{+}或\texttt{+=} 操作符添加其他的委托(或方法)
\item<2-| handout:1> 通过 \texttt{-}或\texttt{-=} 操作符去掉已有的委托(或方法)
\item<3-| handout:1> 多播委托的返回值应该是 void
\item<3-| handout:1> 添加和去掉方法的格式比较自由
\end{itemize}
\begin{onlyenv}<1-| handout:1>
\begin{lstlisting}[escapeinside=<>]
  delegate void MyOperation (int i);    // <定义委托>
  MyOperation myop = new MyOperation(); // <创建委托>

  myop += SomeMethod;
  myop += new MyOperation(AnotherMethod);
  ...

  myop(100);
  myop -= new MyOperation(SomeMethod);
  myop -= AnotherMethod;

\end{lstlisting}
\end{onlyenv}
\end{frame}

\begin{frame}[fragile]
\frametitle{委托的协变 (\textit{Covariance})}
\begin{lstlisting}
class Dog {}
class Labrador : Dog {}

class Program
{ public delegate Dog HandlerMethod();

  public static Dog FirstHandler()  
    { ...; return aDog; }
  public static Labrador SecondHandler()
    { ...; return aLabrador; }

  static void Main() 
    {
      HandlerMethod handler1 = FirstHandler;
      HandlerMethod handler2 = SecondHandler;
    }
}
\end{lstlisting}
\end{frame}

\begin{frame}[fragile]
\frametitle{委托的逆变 (\textit{Contravariance})}
\begin{lstlisting}
class Dog {}
class Chiwawa : Dog {}
 
class Program
{ public delegate void HandlerMethod(Chiwawa c);
 
  public static void FirstMethod(Dog dog)
    {...}
  public static void SecondMethod(Chiwawa c)
    {...}
 
  static void Main()
    {
      HandlerMethod handler1 = FirstMethod;
      HandlerMethod handler2 = SecondMethod;
    }
}
\end{lstlisting}
\end{frame}

\begin{frame}[fragile]
\frametitle{委托的逆变(示例2)}
\begin{lstlisting}
System.DateTime lastActivity;
public Form1() // Form1 Constructor
{ InitializeComponent();
  lastActivity = new System.DateTime();

  //works with KeyEventArgs
  this.textBox1.KeyDown   += this.MultiHandler;

  //works with MouseEventArgs
  this.button1.MouseClick += this.MultiHandler;
}
private void
MultiHandler(object sender, System.EventArgs e)
{
  lastActivity = System.DateTime.Now;
}
\end{lstlisting}
\end{frame}

\begin{frame}
\frametitle{Covariance and Contravariance}
\begin{itemize}
\setlength{\itemsep}{8pt plus 1pt}
\item Covariance permits a method to have a {\redwarn more derived} return type
  than what is defined in the delegate.
\begin{itemize}
\item 返回类型可以不同
\item 方法的返回类型是委托的派生类
\end{itemize}
\item Contravariance permits a method with parameter types that are
  {\redwarn less derived} than in the delegate type.
\begin{itemize}
\item 参数类型可以不同
\item 方法的参数类型是委托的基类
\end{itemize}
\end{itemize}
\end{frame}

\begin{frame}[fragile]
\frametitle{匿名方法}
\begin{block}{\textit{anonymous method}}
  \CJKindent 匿名方法就是可以执行的一段代码块,没有方法名,只能通过委托调用。
\end{block}
匿名方法的一般格式:
\begin{lstlisting}
delegate [(parameter-list)] { anonymous-method-block}
\end{lstlisting}
\begin{itemize}
\item C\# 2.0 中使用委托可以创建匿名方法
\item 使用匿名方法,可以简化程序,不用为简单的操作定义方法
\end{itemize}
\end{frame}
\begin{frame}[fragile]
\frametitle{匿名方法示例}
\begin{lstlisting}[escapeinside='']
using System;
// '定义委托'
delegate void CountIt();

class AnonMethDemo {

  public static void Main() {

    // '添加一个匿名方法'
    CountIt count = delegate {
      for(int i=0; i <= 5; i++)
        Console.WriteLine(i);
    }; // '注意这里的分号'

    count();
  }
}
\end{lstlisting}

\end{frame}

\begin{frame}[fragile]
\frametitle{带参数的匿名方法}
\begin{lstlisting}[escapeinside='']
using System;
// '定义带参数的委托'
delegate void CountIt(int end);

class AnonMethDemo2 {

  public static void Main() {

    // '添加一个匿名方法'
    CountIt count = delegate (int end) {
      for(int i=0; i <= end; i++)
        Console.WriteLine(i);
    };

    count(5);
  }
}
\end{lstlisting}

\end{frame}

\begin{frame}[fragile, plain]
\frametitle{带返回值的匿名方法}
\begin{lstlisting}
using System;
delegate int CountIt(int end); // return int

class AnonMethDemo3 {
  public static void Main() {
    int result;

    CountIt count = delegate (int end) {
      int sum = 0;

      for(int i=0; i <= end; i++) {
        Console.WriteLine(i);
        sum += i;
      }
      return sum; // here
    };

    result = count(3);
    Console.WriteLine("Summation of 3 is " + result);
  }
}
\end{lstlisting}
\end{frame}

\begin{frame}[fragile]
\frametitle{委托的相等性}
\begin{itemize}
\item 不同的委托间,判断是否已经调用某个方法,需要判断委托的相等性
\item 对于单个委托,只需判断类型和委托中的方法
\item 对于多播委托,还需要判断方法个数和顺序
\end{itemize}
\begin{lstlisting}
  StrMod strOp1;
  StrMod strOp2;
  ...

  bool equal = (strOp1 == strOp2);

\end{lstlisting}
\end{frame}



\section{事件}

\begin{frame}[fragile]
\frametitle{事件}
\begin{block}{\textit{events}}
  \CJKindent 类中安全的委托成员,Windows 编程中的事件在 C\# 语言中的体现,是
  程序元素之间通信的动态格式。
\end{block}
事件定义的一般格式:
\begin{lstlisting}
[modifier] event <delegatetype> <eventname>;
\end{lstlisting}
\pause
\begin{itemize}
\item 事件不是从程序的一部分到另一部分的过程流
\item 只是建立在程序之间的连接的方法和运行过程中的终结操作
\item 程序运行期间,事件用于通知感兴趣的客户
\item 用于 C\# 的组件,提供回调功能
\end{itemize}
\end{frame}

\begin{frame}[fragile]
\frametitle{定义事件}
\begin{lstlisting}
using System;
public delegate void MenuHandler();

public class MenuItem
{
  public event MenuHandler MenuSelection;
  // event: MenuSelection
  string text;
  public MenuItem (string text)
  { this.text = text; }

  public void Fire()
  { MenuSelection(); }

  public string Text
  { get {return text;}
    set {text = value;} }
}
\end{lstlisting}

\end{frame}

\begin{frame}
\frametitle{发布/订阅模式}
\begin{itemize}
\item 提供事件的类可以称为事件的提供器或发布器 (\textit{publisher})
\item 需要被通知的组件可以称为订阅器 (\textit{subscriber})
\end{itemize}
\bigskip \pause
\begin{itemize}
\setlength{\itemsep}{4pt plus 1pt}
\item 订阅器通过向事件中添加事件处理方法来达到订阅的目的
\item 因为事件一般都为 \texttt{public},因此任何组件都可以订阅
\item 而因为委托可以添加多个方法,所以同一时间可由多个组件订阅
\item 而事件发布器通过调用委托,来触发 (\textit{fire}) 事件
\item 事件的触发实际就是委托调用的过程
\end{itemize}
\end{frame}

\begin{frame}[fragile]
\frametitle{委托和事件}
为什么不直接使用委托呢?
\begin{itemize}
\setlength{\itemsep}{6pt plus 1pt}
\item 委托用于事件时,需要声明为 \texttt{public},其他组件才能订阅
\item 但同时也可以触发,即委托的调用,不够安全
\item 因此 C\# 将委托包装为事件,对外专门用于订阅
\item 定义为事件后,其他组件也只能订阅和删除
\item 而作为委托的调用,只能由发布器类自己完成
\end{itemize}
\end{frame}

\begin{frame}[fragile,plain]
\frametitle{事件访问器}
\CJKindent 添加方法到事件,或从事件中删除的过程是可以定制的,如通过{\redwarn 事
件访问器}添加和删除事件。
\lstset{emph={add,remove}, emph={[2]MenuSelection}}
\begin{lstlisting}
public delegate void
  MenuHandler(object sender, EventArgs e);

public class MenuItem
{ int numberOfEvents;
  strng text;

  private MenuHandler mh = null;  // delegate

  public event MenuHandler MenuSelection 
  { // event accessor
    add { mh += value; numberOfEvents++; }
    remove {
          mh -= value; numberOfEvents--; }
  }
}

\end{lstlisting}
\end{frame}

\begin{frame}[fragile]
\frametitle{事件的其他特性}
\setlength{\itemsep}{6pt plus 1pt}
\begin{itemize}
\item 事件可以在接口中,实现接口的类必须实现该事件
\item 事件可以是抽象 (\texttt{abstract}) 的,由派生类继承并实现
\item 事件访问器不可以是抽象的
\item 事件可以被密封 (\texttt{sealed}),防止被继承
\item 事件可以是虚的 (\texttt{virtual}),允许派生类的覆盖(\texttt{override})
\end{itemize}
\end{frame}

\begin{frame}[fragile]
\frametitle{.NET 中的事件}
\begin{itemize}
\item 定义事件,可以使用任意的返回值类型和参数类型
\item 但 .NET 框架推荐了一种标准,即:
\begin{lstlisting}
void Handler (object source, EventArgs arg){
  // ...
}
\end{lstlisting}
\end{itemize}
\medskip
\begin{itemize}
\item source 用于调用方法把自己 (\texttt{this}) 传给客户
\item arg 是 System.EventArgs 类或其派生类型的参数
\item 自定义的 EventArgs 可存放任何信息
\end{itemize}

\end{frame}

\begin{frame}[fragile]
\frametitle{.NET 事件示例(1)}
\begin{lstlisting}
using System;
class MyEventArgs : EventArgs { public int eventnum; }

delegate void MyEventHandler(object source, MyEventArgs arg);

class MyEvent {   static int count = 0;

  public event MyEventHandler SomeEvent;

  public void OnSomeEvent() {   // This fires SomeEvent.
    MyEventArgs arg = new MyEventArgs();

    if(SomeEvent != null) {
      arg.eventnum = count++;
      SomeEvent(this, arg);
    }
  }
}
\end{lstlisting}
\end{frame}

\begin{frame}[fragile]
\frametitle{.NET 事件示例(2)}
\begin{lstlisting}
class X {
  public void handler(object s, MyEventArgs arg) {..} }
class Y {
  public void handler(object s, MyEventArgs arg) {..} }

class EventDemo6 {
  public static void Main() {
    X ob1 = new X();
    Y ob2 = new Y();
    MyEvent evt = new MyEvent();

    evt.SomeEvent += ob1.handler; // Add to the list.
    evt.SomeEvent += ob2.handler;

    evt.OnSomeEvent(); // Fire the event.
    evt.OnSomeEvent();
  }
}
\end{lstlisting}
\end{frame}

\begin{frame}[fragile]
\frametitle{事件与匿名方法}
\begin{lstlisting}
using System;
delegate void MyEventHandler();

class MyEvent {
  public event MyEventHandler SomeEvent;
  ...
  public void OnSomeEvent() {
    if(SomeEvent != null) SomeEvent();
  }
}
...
  MyEvent evt = new MyEvent();

  evt.SomeEvent += delegate {
    Console.WriteLine("Event received.");
  };

  evt.OnSomeEvent();
  evt.OnSomeEvent();

\end{lstlisting}

\end{frame}

\section{泛型}

\begin{frame}[fragile]
\frametitle{泛型}
\begin{block}{\textit{Generics}}
  \CJKindent 泛型就是类型参数化,可以将类型作为参数,根据需要改变类中成员的类型。
  泛型是 .NET 2.0 的新特性。
\end{block}
\pause
\begin{itemize}
\item 通过范型,可以保证更好的类型安全
\item 更高效的重用代码,减少重复开发和代码拷贝
\end{itemize}
\pause
泛型类定义的一般格式
\lstset{emph={<,>}}
\begin{lstlisting}
class identifier<type-param-list>
{ ... }
\end{lstlisting}
泛型类使用的一般格式
\begin{lstlisting}
identifier<type-arg-list>  varname =
  new identifier<type-arg-list>(cons-arg-list);
\end{lstlisting}
\end{frame}


\begin{frame}[fragile]
\frametitle{泛型的例子}
\begin{lstlisting}
class Gen<T> {
  T ob;

  public Gen(T o) { ob = o; }
  public T getob() { return ob; }
  public void showType() {
    Console.WriteLine("Type of T is " + typeof(T));
  }
}
...
   Gen<int> iOb = new Gen<int>(102);

   Gen<string> strOb;
   strOb = new Gen<string>("Generics add power.");
\end{lstlisting}
\end{frame}


\begin{frame}[fragile]
\frametitle{约束条件}

\CJKindent 使用约束条件,可以限制泛型中使用的类型,可以通知编译器可用的成员。

类型约束的一般格式:
\lstset{emph={where}}
\begin{lstlisting}
class classname<type_param>
  where type_param : constraints
{ ... }
\end{lstlisting}
例如
\begin{lstlisting}
class Test<T> where T : A {
  T obj;

  public Test(T o) {
    obj = o;
  }
}
\end{lstlisting}

\end{frame}


\begin{frame}[fragile]
\frametitle{约束的种类}
\begin{itemize}
\setlength{\itemsep}{8pt plus 1pt}
\item 基类约束:在条件表达式中指定基类名字
\item 接口约束:在条件表达式中指定一个或多个接口名字
\item 构造函数约束:无参数的构造函数约束,使用 \texttt{new()}
\item 引用类型约束:使用 \texttt{class},限制为引用类型
\item 值类型约束:使用 \texttt{struct},限制为值类型
\end{itemize}

\end{frame}


\begin{frame}[fragile]
\frametitle{基类约束}
\begin{itemize}
\item 基类约束,使泛型类可以访问基类成员,提高了可用性
\item 只有实现了基类的类才有效,保证了类型安全
\end{itemize}
\begin{lstlisting}
class A {
  public void hello() {
    Console.WriteLine("Hello");
  }
}
class B : A { }

class Test<T> where T : A {
  T obj;
  ...
  public void sayHello() {
    // OK to call hello() because it's declared
    // by the base class A.
    obj.hello();
  }
}

\end{lstlisting}
\end{frame}


\begin{frame}[fragile]
\frametitle{接口约束}
接口约束可以有多个
\begin{lstlisting}
public interface A { ... }
public interface B { ... }

class Test<T> where T : A, B {

  ...

  }
}
\end{lstlisting}
\end{frame}

\begin{frame}[fragile]
\frametitle{构造函数约束}
\begin{lstlisting}
class A<T> // where T:new()
{
  T a;
  public A() {
    a = new T();
    // compiling error
  }
}
class test
{
  static void Main()
  {
    A<int> x = new A<int>();
  }
}
\end{lstlisting}
\end{frame}

\begin{frame}[fragile]
\frametitle{条件表达式的其他用法}
表示类型之间的继承关系
\begin{lstlisting}
class A     { ... }
class B : A { ... }

class Gen<T, V> where V : T {...}
...
  Gen<A, B> x = new Gen<A, B>;
\end{lstlisting}
\pause
多条约束
\begin{lstlisting}
class Gen<T>
  where T : MyClass, IMyInterface, new()
{ }

class Gen<T, V> where T : class
                where V : struct
{ }

\end{lstlisting}
\end{frame}

\begin{frame}[fragile]
\frametitle{自动缺省值}
引用类型和值类型的缺省值分别为 \texttt{null} 和 \texttt{0},如果在泛型中初始
化:
\begin{lstlisting}[escapeinside='']
class Test<T> {   T obj;
  ...
  public Test() {
    obj = null; // '如果是引用类型'
    obj = 0;    // '如果是值类型'
  }
}
\end{lstlisting}

使用 \texttt{default},自动指定
\begin{lstlisting}
...
  obj = default(T);
\end{lstlisting}
\end{frame}


\begin{frame}[fragile]
\frametitle{泛型结构体}
泛型结构体和泛型类类似:
\begin{lstlisting}
struct XY<T> {
  T x;
  T y;
  public XY(T a, T b) {
    x = a;
    y = b;
  }
}
\end{lstlisting}
也可以有约束:
\begin{lstlisting}
struct XY<T> where T : struct {
// ...
}
\end{lstlisting}
\end{frame}

\begin{frame}[fragile]
\frametitle{泛型方法}
\begin{itemize}
\item 泛型类中的方法可以使用类的形参,自动成为泛型方法
\item 普通类中,也可以定义泛型方法
\end{itemize}
\begin{lstlisting}
class ArrayUtils {
  //...
  public static bool
  copyInsert<T>(T e, int idx, T[] src, T[] target) {

    for(int i=0, j=0; i < src.Length; i++, j++) {
      if(i == idx) { target[j] = e;   j++; }
      target[j] = src[i];  }  // end for
  return true;
  }
}
\end{lstlisting}
泛型方法也可以指定约束条件
\begin{lstlisting}
... copyInsert<T>(T e, ... ) where T : class {...}
\end{lstlisting}
\end{frame}

\begin{frame}[fragile]
\frametitle{泛型方法的使用}
使用时可以隐式指定泛型类型
\begin{lstlisting}
  int[] nums = { 1, 2, 3 };
  int[] nums2 = new int[4];

  ArrayUtils.copyInsert(99, 2, nums, nums2);

  string[] strs = { "Generics", "are", "powerful."};
  string[] strs2 = new string[4];

  ArrayUtils.copyInsert("in C#", 1, strs, strs2);
\end{lstlisting}
\pause 当不能自动判断类型时,需要明确指出
\begin{lstlisting}
// class B:A ...
B[] b = { new B(), new B(), new B() };
A[] a = new A[4];

ArrayUtils.copyInsert<A>(new A(), 1, b, a);
\end{lstlisting}
\end{frame}

\begin{frame}[fragile]
\frametitle{泛型委托和泛型接口}
泛型委托
\begin{lstlisting}
delegate T SomeOp<T>(T v);
...
  SomeOp<int> intDel = sum;
\end{lstlisting}
泛型接口
\begin{lstlisting}[escapeinside='']
public interface ISeries<T> {
  T getNext();
  void setStart(T v);
}
// '继承了泛型接口的类,必须是泛型类'
class ByTwos<T> : ISeries<T> {
  T start;
  T val;
  // ...
}
\end{lstlisting}
\end{frame}

\begin{frame}[fragile]
\frametitle{泛型的限制}
\begin{itemize}
\setlength{\itemsep}{8pt plus 1pt}
\item 特性、操作符、索引器和事件,没有泛型形式,但可以在泛型类中使用类型参数
\item \texttt{extern} 不能修饰泛型方法,即泛型方法不能外部实现
\item 指针类型不能用于泛型的参数
\item 静态成员只相应于具体的泛型类型,而不是泛型类本身
\begin{lstlisting}
class Gen<T> {
  public static T x;
}
...
    Gen<int>.x    = 10;
    Gen<double>.x = 5.5;
    Gen<Circle>.x = new Circle();
\end{lstlisting}
\end{itemize}
\end{frame}

% Local Variables:
% mode: LaTeX
% TeX-master: "part-03.tex"
% TeX-header-end: "% End-of-Header$"
% TeX-trailer-start: "% Start-of-Trailer$"
% coding: utf-8
% End:
