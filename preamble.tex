%% Slides for ".NET Programming" by Chunyu Wang <chunyu@hit.edu.cn>

\documentclass[13pt,cjk,xcolor=svgnames,table,serif]{beamer}

\mode<presentation> {%
  \usetheme{Warsaw}
  \setbeamertemplate{background canvas}[vertical shading][bottom=red!10,top=blue!10]
  \setbeamercovered{transparent}
  \setbeamertemplate{headline}{%
    \begin{beamercolorbox}{section in head/foot}
      \vskip2pt\insertnavigation{\paperwidth}\vskip2pt
    \end{beamercolorbox}}%
  \usefonttheme{serif}
  \setbeamertemplate{background}[grid][step=.5cm] 
}

\usepackage{CJK,ccmap}
\usepackage{times}
\usepackage{listings}
\usepackage{pgf}
\usepackage{tikz}
\usepackage{pgflibraryshapes}
\usepackage{pgflibrarytikztrees}
\usepackage[latin1]{inputenc}
\usepackage[english]{babel}

%% C# language for listings
\lstdefinelanguage{CSharp}
{morekeywords={abstract, as, base, bool, break, byte, case, catch, char,
    checked, class, const, continue, decimal, default, delegate, do, double,
    else, enum, event, explicit, extern, false, finally, fixed, float, for,
    foreach, get, goto, if, implicit, in, int, interface, internal, is, lock,
    long, namespace, new, null, object, operator, out, override, params,
    private, protected, public, readonly, ref, return, sbyte, sealed, set,
    short, sizeof, stackalloc, static, string, struct, switch, this, throw,
    true, try, typeof, uint, ulong, unchecked, unsafe, ushort, using, value,
    virtual, void, volatile, while},
  otherkeywords={:},
  sensitive=true,
  morecomment=[l]{//},
  %morecomment=[l]{///},
  morecomment=[s]{/*}{*/},
  morestring=[b]"
}

%% listings setup
\lstset{basicstyle=\ttfamily\small,
  keywordstyle=\color[rgb]{0.15,0.25,0.55}\bfseries,
  identifierstyle=,
  stringstyle=\color[rgb]{0.41,0.13,0.55}\ttfamily,
  commentstyle=\color[rgb]{0.55,0.14,0.14}\sl,
  emphstyle=\color[rgb]{0.75,0,0}\bfseries,
  emphstyle={[2]\color[rgb]{0,0.55,0}\bfseries},
  frameround=tttt,
  frame=single,
  framerule=1pt,
  rulecolor=\color[rgb]{0.18,0.55,0.34},
  backgroundcolor=\color{red!10}, %\color[rgb]{0.94,0.85,0.71},
  escapebegin=\small\CJKfamily{li}\sl\color[rgb]{0.55,0.14,0.14},
  escapeend=,
  language=CSharp}

%% hyperref setup
\hypersetup{%
  colorlinks=false,
  %bookmarks=true,
  bookmarksopen=true,
  pdfstartview=FitV,
  pdfauthor={Chunyu Wang <chunyu@hit.edu.cn>},
  pdftitle={Microsoft .NET Framework},
  pdfsubject={.NET Programming},
  pdfkeywords={.NET, C\#},
  pdffitwindow=true,
  pdfpagemode=None,
  % pdfpagemode=FullScreen,
  % pdfmenubar=false,
  % pdftoolbar=false,
}

\graphicspath{{figures/}}

%% logo definition
\pgfdeclaremask{hit-mask}{logo/hit-mask}
\pgfdeclareimage[mask=hit-mask,width=1.3cm]{hit-logo}{logo/hit-logol}
\pgfdeclaremask{cs-mask}{logo/cs-mask}
\pgfdeclareimage[mask=cs-mask,width=1.3cm]{cs-logo}{logo/cs-logo}
\logo{\vbox{\hbox to 1.3cm{\hfil\pgfuseimage{cs-logo}}\vskip0.1cm\hbox{\pgfuseimage{hit-logo}}}}

%% Slides for ".NET Programming" by Chunyu Wang <chunyu@hit.edu.cn>
%% $Rev$ $LastChangedDate$

\newcommand{\song}{\CJKfamily{song}}
\newcommand{\fs}{\CJKfamily{fs}}
\newcommand{\kai}{\CJKfamily{kai}}
\newcommand{\hei}{\CJKfamily{hei}}
\newcommand{\li}{\CJKfamily{li}}
\newcommand{\you}{\CJKfamily{you}}

\def\helplines{%
  \draw[->] (0,0)--(9,0);  \draw[->] (0,0)--(0,7.2);
  \draw[step=1cm,color=blue!40,very thin] (1pt,1pt) grid (8.9,6.9);
  \draw[step=1cm,color=black!40,dotted,very thin,shift={(-.5,-.5)}] (1pt+.5cm,1pt+.5cm) grid (9.5,7.5);
  \foreach \x in {1,2,...,7} \draw[font=\scriptsize] 
  (\x,0) -- +(up:1pt) ++(down:4pt) node {$\x$}  
  (0,\x) -- +(right:1pt) ++(left:3pt) node {$\x$};}

\def\wraphere{\tikz \draw[color=red!60!black,thick,->] (0ex,1.2ex) .. controls +(-10:1.2ex) and +(40:.2ex) .. (0,0);\\}

\def\cmd#1{\par\texttt{C:$\backslash> $\ {}#1}\par}

% Local Variables: 
% mode: TeX
% coding: gb2312
% End:


% Local Variables: 
% mode: LaTeX
% End:
