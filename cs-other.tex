%% Slides for ".NET Programming" by Chunyu Wang <chunyu@hit.edu.cn> %%

\section{其他}
% Chapter 19 in "C# 2.0 - The Complete Reference, 2nd Edition"

\begin{frame}[fragile]
\frametitle{不安全代码 ~\textit{Unsafe Code}}
% unsafe, pointers

\begin{itemize}
\item 指不受 ~CLR 完全控制的代码 (\textit{unmanaged code})
\item 不安全代码中,无法保证类型安全
\item 对内存操作,需要使用指针 (\textit{pointers})
\item 编译带有不安全代码的程序,需要使用 ~\texttt{/unsafe} 选项
\end{itemize}

C\# 中的指针
\begin{itemize}
\item 一般形式:\texttt{type* varname;}
\item 只有值类型的指针
\item 指针指向的地址可以是任意的
\end{itemize}
\end{frame}

\begin{frame}[fragile]
\frametitle{指针的使用}
\begin{itemize}
\item 指针的声明
\begin{lstlisting}[escapeinside=<>]
int*    i;
float*  p1, p2;   // <注意,p1, p2 都是指针>
int    *p1,*p2;   // <错误,C\#中无效的语法>
\end{lstlisting}

\item 指针的使用
\begin{lstlisting}[escapeinside=<>]
int*  ip;
int   num = 10;
ip  = &num; // <取地址>
num = *ip;  // <访问值>
\end{lstlisting}

\item 指针的类型
\begin{lstlisting}[escapeinside=<>]
int*   p   // <指向整数的指针>
int**  p   // <指向整数的指针的指针>
int*[] p   // <指向整数的指针的一维数组>
char*  p   // <指向字符的指针>
void*  p   // <指向未知类型的指针>
\end{lstlisting}
\end{itemize}
\end{frame}


\begin{frame}[fragile]
\frametitle{不安全代码的使用}
使用关键字 ~\texttt{unsafe} 标记语句或方法
\lstset{emph={unsafe}}
\begin{lstlisting}
using System;
class example {
  unsafe public static void Main() {
    int count = 99;
    int* p;

    p = &count;

    Console.WriteLine("Initial value of count is " + *p);

    *p = 10;

    Console.WriteLine("New value of count is " + *p);
  }
}
\end{lstlisting}
\end{frame}

\begin{frame}[fragile]
\frametitle{\texttt{fixed} 表达式}
为防止 ~GC 改变对象地址,可使用 ~\texttt{fixed} 进行限制
\lstset{emph={fixed}}
\begin{lstlisting}
class Test {
  public int num;
  public Test(int i) { num = i; }
}
...
unsafe public static void Main() {
  Test o = new Test(19);

  fixed (int* p = &o.num) {
    Console.WriteLine("Initial value of o.num is " + *p);
    *p = 10; // assign to o.num via p
    Console.WriteLine("New value of o.num is " + *p);  }
}
\end{lstlisting}
\end{frame}

\begin{frame}[fragile]
\frametitle{指针的运算和比较}
指针移动:\texttt{$++$, $--$, $+$, $-$}
\begin{lstlisting}
int* pl;
pl++;          // Move 4 byte (sizeof(int)=4)
pl = pl + 1;   // Move 4 byte too
\end{lstlisting}
\pause
指针比较:\texttt{$==$,$<$,$>$,$>=$,$<=$}
\begin{lstlisting}
  int[] nums = new int[11];
  int x;

  fixed (int* start = &nums[0]) {
    fixed(int* end = &nums[nums.Length-1]) {
      for(x=0; start+x <= end-x; x++) ;
    }
  }

\end{lstlisting}
\end{frame}

\begin{frame}[fragile]
\frametitle{可空值类型}
当需要值类型也可以 ~\texttt{null} 时,使用可空值类型

\begin{lstlisting}[escapeinside='']
System.Nullable<int> count = null;
System.Nullable<bool> done;

int? count;
bool? done = null;

...

if (count != null) { ... }
if (count.HasValue) { ... } // '可空值的特性'
\end{lstlisting}

\CJKindent 在处理数据库和其他包含不可赋值的元素的数据类型时,将 null 赋值给数值
类型或布尔型的功能特别有用。例如,数据库中的布尔型字段可以存储
值 true 或 false,或者,该字段也可以未定义。
\end{frame}

\begin{frame}[fragile]
\frametitle{可空值类型}
\begin{itemize}
\item 可空值类型表示可被赋值为~\texttt{null} 值的值类型变量
\item 语法~\texttt{T?} 是~\lstinline|Nullable<T>| 的简写,\texttt{T} 为值类型,两种形式可以互换
\item 使用~\texttt{HasValue} 和~\texttt{Value} 只读属性测试是否为空和检索值
\begin{itemize}
\item 布尔值~\texttt{HasValue} 表示是否包含有效值
\item 如果已赋值,则~\texttt{Value} 属性返回该值,否则将引发异常 System.InvalidOperationException 
\end{itemize}
\end{itemize}
\end{frame}

\begin{frame}[fragile]
\frametitle{空值接合操作符 (\textit{null coalescing operator})}
当可空值为空时,操作符取其后的默认值:
\begin{lstlisting}
double? balance = null;
double  currentBalance;

currentBalance = balance ?? 0.0;
\end{lstlisting}

当可空值不为空时,取变量自身的值:
\begin{lstlisting}
double? balance = 123.75;
double currentBalance;

currentBalance = balance ?? 0.0;
\end{lstlisting}
\end{frame}

\begin{frame}[fragile]
\frametitle{部分类 (\textit{partial class})}
使用 ~\texttt{partial} 修饰,可以将类的定义拆分到多个文件中
\lstset{emph={partial}}
\begin{lstlisting}
partial class XY {
  public XY(int a, int b) {
    x = a; y = b;
  }
}
\end{lstlisting}
\lstset{emph={partial}}
\begin{lstlisting}
partial class XY {
  int x, y;
  public int X {
    get { return x; }
    set { x = value; }
  }
  public int Y {
    get { return y; }
    set { y = value; }
  }
}

\end{lstlisting}
\end{frame}

\begin{frame}[fragile]
\frametitle{部分方法 (\textit{partial method})}
\begin{itemize}
\item 定义在部分类中,如果没有实现则不会被编译
\item 必须是私有的,不能返回值,不能有输出参数
\item 常用于处理轻量级事件
\end{itemize}
\lstset{emph={partial}}
\lstset{emph={[2]VerifyDuplicates}}
\begin{lstlisting}[escapeinside=<>]
partial class CarLocator
{
  public bool CarAvailableInZipCode(string zipCode)
  {
    VerifyDuplicates(zipCode);
    <$\ldots$>
    return true;
  }

  partial void VerifyDuplicates(string make);
}
\end{lstlisting}
\end{frame}


\begin{frame}[fragile]
\frametitle{\texttt{extern} 关键字}
关键字 ~\texttt{extern} 声明外部方法,创建外部程序及别名
\cmd{csc /t:library test1.cs}
\cmd{csc /r:Asm1=test1.dll test2.cs}

\begin{lstlisting}
// extern alias statements must be at the top of the file.
extern alias Asm1;

using System;
class Demo {
  public static void Main() {
    Asm1::MyNS.MyClass t = new Asm1::MyNS.MyClass();
  }
}
\end{lstlisting}

\end{frame}

\begin{frame}[fragile]
\frametitle{\texttt{lock} 和~ \texttt{volatile}}
\begin{itemize}
\item 关键字 ~\texttt{lock} 用于多线程执行时,进入临界区
\begin{lstlisting}
lock(obj) { /* critical section */ }
\end{lstlisting}
\item 关键字 ~\texttt{volatile} 表示字段可能被多个并发执行线程修改
\begin{lstlisting}
public volatile int i;
\end{lstlisting}
\end{itemize}
% using Statement
\end{frame}

% Local Variables:
% mode: LaTeX
% TeX-master: "part-03.tex"
% TeX-header-end: "% End-of-Header$"
% TeX-trailer-start: "% Start-of-Trailer$"
% coding: utf-8
% End:

