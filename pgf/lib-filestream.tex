%% Slides for ".NET Programming" by Chunyu Wang <chunyu@hit.edu.cn> %%

\begin{tikzpicture}[font=\footnotesize]
\tikzstyle{every node}=[anchor=west];
\node at (.3,4.7) {FileStream(file name, FileMode)};
\node at (.3,4.35) {FileStream(file name, FileMode, FileAccess)};
\node at (.3,4) {FileStream(file name, FileMode, FileAccess, FileShare)};
\draw (3.5,3.5) node (a1) {Append};
\draw (3.5,3.15) node (a2) {Create};
\draw (3.5,2.8) node (a3) {Open};
\draw (3.5,2.45) node (a4) {CreateNew};
\draw (3.5,2.1) node (a5) {OpenOrCreate};
\draw (3.5,1.75) node (a6) {Truncate};

\draw (5.1,3.5) node (b1) {Read};
\draw (5.1,3.15) node (b2) {\redwarn ReadWrite};
\draw (5.1,2.8) node (b3) {Write};

\draw (6.85,3.5) node (c1) {\redwarn None};
\draw (6.85,3.15) node (c2) {Read};
\draw (6.85,2.8) node (c3) {ReadWrite};
\draw (6.85,2.45) node (c4) {Write};

\foreach \x in {1,2,...,6}
\draw[thick] (3.4,3.8) |- (a\x);
\foreach \x in {1,2,3}
\draw[thick] (5,3.8) |- (b\x);
\foreach \x in {1,2,3,4}
\draw[thick] (6.75,3.8) |- (c\x);

\draw (.3,2.75) node {����ö�����ij�Ա};
\end{tikzpicture}
