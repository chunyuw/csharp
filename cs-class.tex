%% Slides for ".NET Programming" by Chunyu Wang <chunyu@hit.edu.cn> %%

\part{类}
\begin{frame}
\frametitle{Outline}            % {C\# 的类}
\tableofcontents
\end{frame}

\section{类的基本组成}

\begin{frame}[fragile]
\frametitle{类的定义}
类定义的一般格式:
\begin{lstlisting}
[attribute]
[modifiers] class identifier [:baselist]
{
  // class body
}[;]
\end{lstlisting}
\begin{itemize}
\item attribute --- 类的属性
\item modifiers --- 类的修饰符
\item baselist  --- 基类和实现的接口
\end{itemize}
\end{frame}

\begin{frame}[fragile]
\frametitle{一个简单的例子}
\lstset{emph={new}}
\begin{lstlisting}
class Circle {
  public int R;
  public static double pi = 3.14159;
  public double Perimeter(){
    return 2 * pi * R;
  }
}
class test {
  static void Main(){
    Circle c = new Circle();
    c.R = 20;
    System.Console.WriteLine(c.Perimeter());
  }
}
\end{lstlisting}
\end{frame}

\note{
  \begin{itemize}
  \item 这里是一个类的例子,也是一个应用程序的例子
  \item class 来定义类,new 关键字创建实例
  \item C\#中,没有全局的函数和变量,都必须放在类中,即使是 Main()函数
  \item Main() 函数必须是 static 的,而且首字母大写
  \end{itemize}}


\begin{frame}[fragile]
\frametitle{类的属性}
\CJKindent 属性(\textit{attribute})是可选的,可以有一个或多个,用逗号分割并放
在中括号中。

\begin{lstlisting}
[ClassDesc(Author="Jack")]
public class XXX { ... }

[Obsolete("This class should no longer be used")]
public class YYY { ... }

\end{lstlisting}
\pause

\begin{itemize}
\item 属性也可以对类的成员起作用
\item 遵循 CLR 的编译器会将属性信息写入元数据中
\item .NET 框架中定义了几百个预定义的属性,功能繁多
\end{itemize}
\end{frame}

\begin{frame}[fragile]
\frametitle{条件属性}
\begin{lstlisting}
#define DEBUG
using System;
using System.Diagnostics;
class AttriTest
{
  [Conditional ("TRACE")]
  public static void ListTrace()
  { Console.WriteLine("Trace is On"); }

  [Conditional ("DEBUG")]
  public static void ListDebug()
  { Console.WriteLine("Debug is On"); }

  public static void Main()
  { ListTrace();
    ListDebug();
    Console.ReadLine(); }
}
\end{lstlisting}
\end{frame}

\begin{frame}[fragile]
\frametitle{访问 DLL}
调用 powrprof.dll 中方法执行系统休眠:
\lstset{emph={DllImport,extern}}
\begin{lstlisting}
using System;
using System.Runtime.InteropServices;

class suspend
{
  [DllImport( "powrprof.dll",
              EntryPoint="SetSuspendState",
              CharSet=CharSet.Ansi )]

  static extern int SetSuspendState(int H, int F, int D);
    // H=Hibernate, F=ForceCritical, D=DisableWakeEvent

  public static void Main(string[] args)
  {
    SetSuspendState(1, 0, 0);
  }
}
\end{lstlisting}
\end{frame}


\begin{frame}
\frametitle{访问修饰符}

{\CJKindent C\# 中通过类或类成员的访问限制,能够完全控制谁能使用类或类的成员。}
\bigskip \pause

\begin{tabular}[tb]{l|l}
  \hline 修饰符      & 访问权限                               \\
  \hline
  private            & 仅用于嵌套类,其容器类可访问           \\
  protected          & 仅用于嵌套类,其容器类和派生类可访问   \\
  protected internal & 仅用于嵌套类,当前程序集或派生类可访问 \\
  internal           & 当前程序集内可访问                     \\
  public             & 任何程序集中都可以访问                 \\
  \hline
\end{tabular}
\bigskip \pause

\begin{itemize}
\item 如果省略,默认的访问权限为 internal
\item 派生类和基类要有相同的访问权限
\end{itemize}
\end{frame}

\begin{frame}[fragile]
\frametitle{类的继承}

  \begin{block}{继承}
    \CJKindent 一个类(派生类)从另一个类(基类)中共享特性和行为的方法。
  \end{block}
\pause

\begin{lstlisting}[escapeinside=<>]
class Circle {
  public int R; // <半径>
  public static double pi = 3.1415;
  public double Area(){
    return pi * R * R;
  }
}
class Ring : Circle {
  public int r; // <内半径>
}
\end{lstlisting}
\end{frame}

\begin{frame}[fragile]
\frametitle{类的继承}
  \begin{itemize}
    \setlength{\itemsep}{8pt plus 1pt}
  \item 建立类的层次结构,公用的功能放在基类\pause
  \item 代码重用,基类的字段和代码\pause
  \item 派生类继承基类的所有成员,但基类可以通过 \texttt{private} 限制派生类访问\pause
  \item C\#只允许单继承,但可实现多个接口(\textit{interface})\par
\lstinline|public class MyClass:Interface1, Interface2 {...}|\pause
  \item 通过\texttt{sealed}修饰的密封类,不能被继承\par
\lstinline|public sealed class secret {...}|
  \end{itemize}
\end{frame}

\begin{frame}
\frametitle{类的成员}
\begin{onlyenv}<1| handout:1>
  \begin{block}{字段 (\textit{fields})}
    描述对象属性的简单数据类型、自定义的结构或其他类。
  \end{block}
  \begin{block}{方法 (\textit{methods})}
    对象中执行计算或其他操作的函数成员。
  \end{block}
  \begin{block}{特性 (\textit{properties})}
    对类中字段提供特定的访问方法。
  \end{block}
\end{onlyenv}
\begin{onlyenv}<2| handout:2>
  \begin{block}{委托 (\textit{delegates})}
    运行时用来动态调用不同的方法的特殊成员。
  \end{block}
  \begin{block}{事件 (\textit{events})}
    建立程序之间的连接方法和运行过程的终结操作。
  \end{block}
  \begin{block}{索引器 (\textit{indexers})}
    通过方括号的语法访问成员的方法。
  \end{block}
\end{onlyenv}
\end{frame}


\begin{frame}[fragile]
\frametitle{静态成员和实例成员}
\begin{onlyenv}<1| handout:1>
  \begin{itemize}
  \item 静态成员使用关键字static修饰,否则为实例成员
\begin{lstlisting}[escapeinside=<>]
class Circle {
  // <实例变量>
  public int R;
  // <静态变量>
  public static double pi = 3.14159;
  // <实例方法>
  public double Perimeter(){
    return 2 * pi * R;
  }
}
\end{lstlisting}

  \end{itemize}
\end{onlyenv}
\begin{onlyenv}<2| handout:2>
  \begin{itemize}
\setlength{\itemsep}{8pt plus 1pt}
  \item 静态成员通过类名访问,实例成员通过对象名访问
\begin{lstlisting}[escapeinside=<>]
Circle c  = new Circle(); // <创建一个实例>
Circle.pi = 3.14159265;   // <访问静态变量>
      c.R = 100;          // <访问实例变量>
\end{lstlisting}
  \item 静态成员和具体实例无关,是类本身的特征
  \item 通过静态变量,可以在类的实例之间共享信息
  \item 静态成员在加载类的时候初始化,分配内存空间
  \end{itemize}
\end{onlyenv}
\end{frame}

\begin{frame}[fragile]
\frametitle{常量和只读变量}
\begin{itemize}
\setlength{\itemsep}{8pt plus 1pt}
\item 分别使用const和readonly修饰
\item 常量在编译时初始化,只读变量在运行时初始化
\item 初始化之后,都不可以被修改
\item 引用类型只能使用只读变量,因为无法在编译时初始化
\end{itemize}
\lstset{emph={readonly,const}}
\begin{lstlisting}
  const    string url1="www.hit.edu.cn";
  readonly string url2="www.cs.hit.edu.cn";

  readonly DateTime createtime = DateTime.Now;
\end{lstlisting}
\end{frame}

\section{类的方法}
\begin{frame}
\frametitle{方法(\textit{methods})}
\begin{itemize}
\setlength{\itemsep}{8pt plus 1pt}
\item 方法是一个命名的代码块,用于计算或其他操作的函数成员,表示类的行为
\item 分为实例方法和静态方法,静态方法通过类名访问
  \begin{itemize}
  \item Array.Sort()
  \item Array.Reverse()
  \end{itemize}
% \item 方法可以被声明为 virtual、abstract 或 sealed
% \begin{itemize}
% \item virtual --- 可以被派生类覆盖
% \item abstract --- 没有具体实现,需要派生类实现
% \item sealed --- 不可以被覆盖
% \end{itemize}
\item 方法可以被继承、覆盖和隐藏
\end{itemize}
\end{frame}

\begin{frame}
\frametitle{方法的修饰符}
\begin{itemize}
    \setlength{\itemsep}{6pt plus 1pt}
\item static --- 静态方法,不是类实例一部分
\item virtual --- 虚方法,可以在子类中被覆盖
\item override --- (只能)用于覆盖基类中的同名的虚方法
\item new --- 用于隐藏基类中的方法,无论是否是虚方法
\item sealed --- 禁止派生类覆盖该方法,同时需要 override 修饰
\item abstract --- 只有方法声明,没有具体实现
\item extern --- 只有方法声明,在外部实现
\end{itemize}
\end{frame}


\begin{frame}[fragile]
\frametitle{构造函数(\textit{constructors})}
\begin{block}{构造函数}
  \CJKindent 构造函数是一个(或多个)特殊的方法,用于初始化对象。可以分为静态构
  造函数和实例构造函数两种。
\end{block}
\pause \medskip
\begin{itemize}
\item 构造函数名字与类名相同,但无返回类型
\begin{lstlisting}[escapeinside=<>]
class Circle { public int R;
  // <初始化为单位圆,实例构造函数>
  public Circle() { R=1;}    }
\end{lstlisting}
  \pause \medskip
\item 静态构造函数(用static修饰),在加载类时调用,用于初始化静态变量
\end{itemize}
\end{frame}
\note{
  \begin{itemize}
  \item 构造函数的定义不是必须的
  \end{itemize}
}

\begin{frame}[fragile]
\frametitle{私有构造函数}
\begin{itemize}
\item 用 private 修饰的构造函数,不允许类以外的成员访问。
\end{itemize}
\begin{lstlisting}
class Conversions{
  static cmPerInch = 2.54;
  private static double gmPerPound = 455;
  public static double inchesToMetric(double inches){ ... }
  public static double poundsToGrams(double pounds){ ... }
  private Constructors() {...}
}
\end{lstlisting}
\pause
\begin{itemize}
\item 因为 private 修饰,无法为类创建实例
\item 一般用于使用静态方法和成员提供服务的类
\pause
\item 比如 System.Math 类
\begin{itemize}
\item System.Math.PI, System.Math.E   --- 静态成员
\item System.Math.Pow(), System.Math.Sin() --- 静态方法
\end{itemize}
\end{itemize}

\end{frame}


\begin{frame}[fragile]
\frametitle{析构函数(\textit{destructors})}
析构函数是当对象被清除时被调用的特殊方法。
\begin{itemize}
\item 名字:以字符“\verb||”开始的类名,无返回类型
\begin{lstlisting}[escapeinside=<>]
class Circle {
  public int R;
  public Circle() { R=1;}
  Circle() { /* <析构函数体> */  }
}
\end{lstlisting}
  \pause
\item 由于 C\# 的自动垃圾回收,一般不需要自己创建析构函数
  \pause
\item 构造函数和析构函数的定义都不是必须的
\end{itemize}
\end{frame}

\begin{frame}[fragile,plain]
\frametitle{构造函数和析构函数示例}
\begin{lstlisting}
using System;
public class Person {
  private static int Count = 0;
  public Person() {
    Count++ ;
    Console.WriteLine("total number is: {0}", Count);
  }
  Person() { if(Count>0) Count--;}
}

class test{
  public static void Main(){
    Person[] aGroup = new Person[10];
    for (int i=0;i<10;i++)
      aGroup[i] = new Person();

    Person  another = new Person();
    Console.ReadLine();
  }
}

\end{lstlisting}
\end{frame}


\begin{frame}[fragile]
\frametitle{重载(\textit{overload})}
\begin{block}{方法的重载}
  \CJKindent 对不同的对象,使用相同的方法进行操作。在同一个类中,定义同一名称的
  多个成员,被重载的成员方法之间的唯一差别就是具有不同的参数类型和个数。
\end{block}
\pause
\begin{itemize}
\item 索引器、构造函数、操作符都可以重载
\item 调用时,C\#会根据参数列表的不同自动选择
\end{itemize}
\pause
\begin{lstlisting}[escapeinside=<>]
class Circle{   public int R;
  public Circle()      { R = 1;}
  public Circle(int x) { R = x;}
}  ...
  Circle a = new Circle();
  Circle b = new Circle(20);
\end{lstlisting}
\end{frame}

\begin{frame}[fragile]
\frametitle{重载操作符(\textit{operator overload})}
C\# 可以对操作符重载,和 C++ 中的类似。一般格式为
\pause
\begin{lstlisting}
public static <return_type> operator <op> (<para_list>)
{ ... }
\end{lstlisting}
\pause
\begin{itemize}
\item 必须使用 public 和 static 修饰
\item 返回类型不能是 void
\item 参数个数要匹配
\end{itemize}
\pause
String 类型之间通过 ``+'' 的连接操作,就是一种重载:
\begin{lstlisting}
string a = "abc";
string b = "efg";
string c = a + b;
\end{lstlisting}
\end{frame}

\begin{frame}[fragile,plain]
\begin{lstlisting}
using System;
class Circle
{ public int R;
  public static Ring operator-(Circle left,Circle right)
  { Ring result = new Ring();
    if(left.R > right.R)
    { result.R = left.R;
      result.r = right.R;
    } else ...
    return result;
  }
}
class Ring:Circle
{ public int r;

  public static void Main()
  {
    Circle a = new Circle(); Circle b = new Circle();
    a.R = 100;   b.R = 50;
    Ring c = a - b;
  }
}
\end{lstlisting}
\end{frame}

\begin{frame}[fragile]
\frametitle{重载 true 和 false}
\begin{lstlisting}
class Point {
  int x, y, z; // 3-D coordinates
  public Point() { x = y = z = 0; }
  public Point(int i, int j, int k)
  { x = i; y = j; z = k; }

  public static bool operator true(Point op) {
    return (op.x!=0) || (op.y!=0) || (op.z!=0);
  }
  public static bool operator false(Point op) {
    return (op.x==0) && (op.y==0) && (op.z==0);
  }
}
...
    Point p = new Point(5, 6, 7);

    if(p)  Console.WriteLine("p is true.");
\end{lstlisting}
\end{frame}

\begin{frame}[fragile]
\frametitle{重载一元操作符}
\begin{lstlisting}
class ThreeD
{ int x, y, z;  }
\end{lstlisting}
\medskip
\begin{lstlisting}
public static ThreeD operator -(ThreeD op)
{
  ThreeD r = new ThreeD();
  r.x = -op.x;  r.y = -op.y;  r.z = -op.z;
  return r;
}
\end{lstlisting}
\medskip
\begin{lstlisting}
public static ThreeD operator ++(ThreeD op)
{
  op.x++;   op.y++;   op.z++;
  return op;
}
\end{lstlisting}
\end{frame}

\begin{frame}[fragile]
\frametitle{转换重载}
\begin{itemize}
\item 转换 (\textit{conversion}) 重载允许自己定义类型间的转换
\item 可以分为隐式和显式重载
\item 对于同一种类型,不能同时存在隐式和显式重载
\end{itemize}
\pause
转换重载的一般形式
\lstset{emph={implicit, explicit}}
\begin{lstlisting}
public static operator implicit <target>(<source> v)
  { ... return value; }
public static operator explicit <target>(<source> v)
  { ... return value; }
\end{lstlisting}
\end{frame}

\begin{frame}[fragile]
\frametitle{转换重载示例}
\begin{onlyenv}<1| handout:1>
隐式重载,用于隐式转换:
\lstset{emph={implicit}}
\begin{lstlisting}
...
  public static implicit operator double(ThreeD p)
  {
    double temp = p.x*p.x + p.y*p.y + p.z*p.z;
    return Math.Sqrt(temp);
  }
...
  ThreeD a = new ThreeD(1,2,3);

  double length = a;

\end{lstlisting}
\end{onlyenv}
\begin{onlyenv}<2| handout:2>
显式重载,用于显式转换:
\lstset{emph={explicit}}
\begin{lstlisting}
...
  public static explicit operator double(ThreeD p)
  {
    double temp = p.x*p.x + p.y*p.y + p.z*p.z;
    return Math.Sqrt(temp);
  }
...
  ThreeD a = new ThreeD(1,2,3);

  double length = (double) a;

\end{lstlisting}
\end{onlyenv}
\end{frame}

\begin{frame}[fragile]
\frametitle{特性(\textit{properties})}
\begin{block}{特性}
\CJKindent 对字段提供特定的访问方式,为类提供更好封装。
\end{block}
\begin{itemize}
\item 特性的访问是透明的,类似成员变量
\item get提供读访问,set提供写访问,可以有不同的访问权限
\item 通过特性可进行完整性检查,类似成员方法
\end{itemize}
\pause
特性的一般格式:
\begin{lstlisting}
[attributes] <modifiter> <data_type> <property_name>
{  [access modifier] get {  ...
     return <property_value>;
   }
   [access modifier] set {  ...  
   }
}
\end{lstlisting}
\end{frame}

\begin{frame}[fragile]
\frametitle{特性(\textit{properties})}
  \lstset{emph={get,set}}
\begin{lstlisting}[escapeinside=`']
class Circle {
  protected int R;
  public  int Radius {
    get { return R; }
    set {
      if(value<0)
        throw new ArgumentOutOfRangeException(
             "Radius must be greater than 0.");
      R = value;    // `set 的参数 value'
    }
  }
}
...
  Circle c = new Circle();
  c.Radius = 20;
  Console.WriteLine(c.Radius);

\end{lstlisting}
\end{frame}

\begin{frame}
\frametitle{特性的使用}
\begin{itemize}
\setlength{\itemsep}{8pt plus 1pt}
\item 特性的修饰符可以是
  \begin{itemize}
  \item static --- 静态的,只和类本身相关
  \item abstract --- 用在抽象类中
  \item new, virtal, override --- 继承与覆盖关系
  \end{itemize}
\item set 中有一个隐式参数 value
\item set 和 get 可以只有一个
\end{itemize}
\end{frame}

% 嵌套类

\begin{frame}[fragile]
\frametitle{索引器 (\textit{indexers})}

\begin{block}{索引器}
\CJKindent \small 使用数组索引方式访问类成员。使用方法和特性类似,但可以有多个
参数。
\end{block}
\begin{itemize}
\item 索引器表现和数组类似,通过索引访问
\item 实现方式和特性类似,可以有 get 和 set 访问器
\item 索引器不可以用 static 修饰,因为只与实例相关
\end{itemize}

索引器的一般格式:
\begin{lstlisting}
[attribute]
<modifier><return_type> this [parameters]
{  ... }
\end{lstlisting}

\end{frame}

\begin{frame}[fragile]
\frametitle{索引器 (\textit{indexers}) 示例}
通过索引器设置成员变量
\begin{lstlisting}
class Ring:Circle{
  private int r; // and R from Circle
  public int this[int index]
  { set {
      if (index == 0 )     r = value;
      else if (index == 1) R = value;
    }
  }
}
...
  public static void Main()
  {
    Ring x = new Ring();
    x[0] = 20;
    x[1] = 40;
  }
\end{lstlisting}
\end{frame}

\begin{frame}[fragile]
\frametitle{索引器的重载}
\begin{lstlisting}
class Ring:Circle{
  private int r; // and R from Circle
  public int this[ int index ]   { ...  }
  public int this[ string name ] // by name
  { set {
      if (name == "inner" )
      {  r = value; }
      else if (name == "outer")
      {  R = value; }
    }
  }
}
...
    Ring x = new Ring();
    x["inner"] = 20;
    x["outer"] = 40;
\end{lstlisting}


\end{frame}

\section{类型转换}

\setbeamerfont{table}{size=\small}
\begin{frame}
\frametitle{所有类的基类 System.Object}
\begin{itemize}
\item C\#中的所有类型直接或间接的基类,也是 .NET Framework 中所有类的最终基类
\item 自定义的类,都对 Object 隐式继承,不需要在类的定义中声明
\end{itemize}
\pause
\begin{block}{为派生类提供的底层服务}
  \usebeamerfont{table}
  \begin{tabular}{l|l}
    Equals()      & 确定两个对象实例是否相等         \\
    GetHashCode() & 生成对象相对应的数字以支持哈希表 \\
    ToString()    & 生成描述类的实例的可读文本字符串 \\
    GetType()     & 获取当前实例的类型               \\
    Finalize()    & 在自动回收对象之前执行清理操作   \\
  \end{tabular}
  \normalfont
\end{block}
\end{frame}


\begin{frame}[fragile]
\frametitle{值类型与引用类型}
\begin{itemize}
\item 值类型,赋值或传递给一个方法时,复制实例的内容
\item 引用类型,赋值或传递给一个方法时,使用当前实例,即仅传引用(地址)
\end{itemize}
\begin{columns}
\column{.6\textwidth}
\begin{lstlisting}[escapeinside=<>]
class Circle { public int R; }
...
  Circle c = new Circle();
  c.R = 10;
  Circle d = c;
  d.R = 20;
  Console.WriteLine(c.R);
  // <输出为> 20
\end{lstlisting}
\column{.4\textwidth}
\begin{figure}
  \centering
  %% Slides for ".NET Programming" by Chunyu Wang <chunyu@hit.edu.cn> %% $Rev$

\begin{tikzpicture}[scale=.6,>=stealth,font=\footnotesize]
\fill[fill=yellow!10,draw=gray,fill opacity=.5, rounded corners] (.5 ,0) rectangle +(7.5,5.5) ;

\node[anchor=center] at (2, 0.5) {�й�ջ};
\node[anchor=center] at (5.75, 0.5) {�йܶ�};

\filldraw[draw=black,fill=blue!20,fill opacity=.4,rounded corners] (1,1) rectangle +(2,4);

\draw (1,4) -- (3,4);
\draw (1,3) -- (3,3);
\draw (1,2) -- (3,2);

\filldraw[draw=black,fill=blue!20,fill opacity=.4,dashed,rounded corners] (4,1) rectangle +(3.5,3.5);

\fill[red!40,rounded corners] (4.75, 2) rectangle +(2,1);
\node[anchor=center, font=\small] at (1.9, 2.5) {c};
\node[anchor=center, font=\small] (d) at (1.9, 3.5) {d};

\draw (5.75, 2.5) node[anchor=center] (a) {Circle};
\draw[->,thick] (2.75,2.5) .. controls (3.45,2.8) and (4.15,2.8) .. (a.west);
\draw[->,thick] (2.75,3.5) .. controls +(right:1cm) and + (145:1cm) .. (a.north west);

\end{tikzpicture}

\end{figure}
\end{columns}
\end{frame}

% \begin{frame}
% \frametitle{类型的层次结构}
% \color[rgb]{1,0,0} 31页层次结构图
% \end{frame}

\begin{frame}[fragile]
\frametitle{值类型和引用类型的区别}

\begin{itemize}
\item 内存分配

  \CJKindent \small 值类型和引用类型之间的最主要差别就是内存分配。如果没有进行
  初始化,引用类型默认为 null,值类型为 0。
\pause
\item 内存释放

  \CJKindent \small 变量超出作用域时,值类型自动从栈中释放,而堆中的则不释放,
  根据垃圾自动回收收集。
\pause
\item 变量的赋值

  \CJKindent \small 值类型之间的赋值,直接复制内容({\redwarn 传值}),而引用类型之间只复制引用值({\redwarn 传引用})。
\pause
\end{itemize}
\begin{lstlisting}[escapeinside=<>]
struct Circle { public int R; }
...
  Circle c = new Circle();
  c.R = 10;
  Circle d = c; // <d 的内存在哪儿>
  d.R = 20;
  Console.WriteLine(c.R);
  // <输出为> 20

\end{lstlisting}

\end{frame}


\begin{frame}[fragile]
\frametitle{类型之间的转换}
\begin{block}{隐式转换}
  \begin{itemize}
  \item 当信息不会丢失时,如 int$\rightarrow$long,int$\rightarrow$float
  \item 从派生类到基类的转换,如 Ring$\rightarrow$Circle,Ring$\rightarrow$object
  \end{itemize}
\end{block}
\pause
\begin{block}{显式转换}
  \begin{itemize}
  \item 当不能进行隐式转换时,必须使用显式转换
  \item 如果不使用显式转换,系统会发生编译错误
  \item 语法类似 C 中的强制类型转换
  \end{itemize}
\end{block}
\pause
\begin{lstlisting}[escapeinside=<>]
  int  a = 5;
  long b = a;         // <隐式转换>
  int  c = (int) b;   // <显式转换>

\end{lstlisting}
\note{转换是指把对象从一种类型改变为另外一种类型的能力。}
\end{frame}

\begin{frame}[fragile]
\frametitle{引用类型之间的转换}
\begin{itemize}
\item 任何类型都可以隐式转换为 object
\item 而再从 object 回到具体类型,需要显式转换
\item 实现了某种接口的类型,可隐式转换为该接口的引用类型
\item 转换后的引用可以调用自己的方法,不过可能已经被覆盖
\end{itemize}
\pause
\begin{lstlisting}
using System;
using System.Text;
class Test {
  static void Main () {
    StringBuilder x = new StringBuilder ("hello");
    object y = x;
    x.Append (" there");
    Console.WriteLine ((StringBuilder)y);
    // prints "hello there"
  }
}
\end{lstlisting}
\end{frame}

\begin{frame}[fragile]
\frametitle{值类型与引用类型之间的转换}
\begin{block}{装箱 (\textit{boxing})}
  \begin{itemize}
  \item 将值类型转换为引用类型的隐式过程
  \item 系统为引用类型分配新内存,创建对象,并把值类型的内容复制到新分配的内存中。
  \end{itemize}
\end{block}
\pause
\begin{block}{拆箱 (\textit{unboxing})}
  \begin{itemize}
  \item 将引用类型的对象转换为值类型的过程
  \item 拆箱过程必须是显式的
  \end{itemize}
\end{block}
\pause
\begin{lstlisting}[escapeinside=<>]
  int    a = 5;
  object o = a;       // <隐式转换,装箱>
  int    b = (int) o; // <显式转换,拆箱>

\end{lstlisting}
\end{frame}

\begin{frame}[fragile]
\frametitle{装箱与拆箱}

\begin{lstlisting}
class Test {
  static void Main () {
    int x = 100;
    object o = x; // boxing
    int y = (int)o; // unboxing
  }
}

\end{lstlisting}
  \begin{figure}[htbp]
    \centering
    %% Slides for ".NET Programming" by Chunyu Wang <chunyu@hit.edu.cn> %% -*- coding: utf-8 -*-

\begin{tikzpicture}[scale=.7,>=stealth,font=\footnotesize]
\fill[fill=yellow!10,draw=gray,fill opacity=.5, rounded corners] (.5 ,0) rectangle +(7.5,5.5) ;

\node[anchor=center] at (2, 0.5) {托管栈};
\node[anchor=center] at (5.75, 0.5) {托管堆};

\filldraw[draw=black,fill=blue!20,fill opacity=.4,rounded corners] (1,1) rectangle +(2,4);

\draw (1,4) -- (3,4);
\draw (1,3) -- (3,3);
\draw (1,2) -- (3,2);

\filldraw[draw=black,fill=blue!20,fill opacity=.4,dashed,rounded corners] (4,1) rectangle +(3.5,3.5);

\fill[red!40,rounded corners] (4.75, 2) rectangle +(2,1);
\node[anchor=center, font=\small] (d) at (1.9, 3.5) {y: 100};
\node[anchor=center, font=\small] at (1.9, 2.5) {o};
\node[anchor=center, font=\small] at (1.9, 1.5) {x: 100};

\draw (5.75, 2.5) node[anchor=center] (a) {100};
\draw[->,thick] (2.75,2.5) .. controls (3.45,2.8) and (4.15,2.8) .. (a.west);
%\draw[->,thick] (2.75,3.5) .. controls +(right:1cm) and + (145:1cm) .. (a.north west);

\end{tikzpicture}

  \end{figure}

\end{frame}

\begin{frame}[fragile]
\frametitle{\texttt{is} 和 \texttt{as} 操作符}
\begin{itemize}
\item is 操作符检查变量是否属于指定类型
\begin{lstlisting}
int i = 0;
bool iIsInt = i is int; // true

Ring r = new Ring();
bool rIsCircle = r is Circle; // true
\end{lstlisting}
\pause
\item as 引用类型的转换,如果转换不成功,赋值为 null
\begin{lstlisting}
Ring   r = new Ring();
Circle o = r as Circle;
Console.WriteLine("r {0} a Circle",
  o == null ? "is not" : "is");
\end{lstlisting}
\end{itemize}
\end{frame}

\begin{frame}[fragile]
\frametitle{\texttt{is} 和 \texttt{as} 操作符 (续)}
\begin{itemize}
\item is 可作用于值类型
\item as 可减少类型检查的次数
\end{itemize}
\begin{lstlisting}
// use is
if (o is SomeThing){
  SomeThing x = (SomeThing) o;
  ...
}

// use as
SomeThing x = o as SomeThing;
if (o != null){
  ...
}

\end{lstlisting}
\end{frame}

\begin{frame}[fragile]
\frametitle{方法的参数传递}
\begin{itemize}
\item 值类型的参数,默认为{\redwarn 传值}
\item 引用类型的参数,默认为{\redwarn 传引用}
\end{itemize}
\begin{columns}
  \column{.5\textwidth}
\lstset{emph={struct}}
\begin{lstlisting}
struct Point {...}

public void Test(Point m){
  m.X = -10; m.Y = -10;
}

  Point p = new Point(10,20);
  Test(p);
  // here p == (10,20);
\end{lstlisting}
  \column{.5\textwidth}
\lstset{emph={class}}
\begin{lstlisting}
class Point {...}

public void Test(Point m){
  m.X = -10; m.Y = -10;
}

  Point p = new Point(10,20);
  Test(p);
  // here p == (-10,-10);
\end{lstlisting}
\end{columns}
\end{frame}

\begin{frame}[fragile]
\frametitle{\texttt{ref} 和 \texttt{out}}
\begin{itemize}
\item out 表示被调用函数通过入口参数,向调用函数返回结果
\item ref 表示被调用函数使用入口参数的引用
\item 调用时候也需明确指出 ref 和 out 参数
\item 明确使用 ref 和 out 可以提高代码可读性
\end{itemize}
\pause
\begin{lstlisting}
static void FillArray (out double[] prices)
{  prices = new double[4] { 10,20,30,40 };  }
static void ChangeValue (ref int x)
{  x = 10;  }

static void Main()
{  double[] p;
   FillArray(out p);

   int y = 20;
   ChangeValue (ref y); // y is 10 now
}
\end{lstlisting}
\end{frame}


\begin{frame}[fragile]
\frametitle{可变长参数}
\begin{itemize}
\item 由 params 关键字修饰的数组参数,使用时是可变长参数
\item 使用 params 可以避免显式的创建数组、传递数组的过程
\item params 只能用于最后一个参数
\end{itemize}
\pause
\begin{lstlisting}
static double Average (string n, params double[] l)
{  double total = 0.0;
   foreach(double x in l)   total += x;
   return total/l.Length;
}
static void Main()
{  double result;
   result = Average("Jack", 80.2, 75.8, 92.3);
   System.Console.WriteLine (result);
   result = Average("Mike", 79.4, 65.8);
   System.Console.WriteLine (result);
}
\end{lstlisting}
\end{frame}

\begin{frame}
\frametitle{结构体和类的比较}
\begin{figure}
  \centering
  \begin{tabular}{l|l|l}
    \hline
                       & 结构体   & 类       \\
    \hline
    类型的默认访问级别 & internal & internal \\
    成员的默认访问级别 & private  & private  \\
    类型               & 值类型   & 引用类型 \\
    能否作为基类       & 不能     & 能       \\
    构造函数           & 有$^{1}$ & 有       \\
    析构函数           & 无       & 有       \\
    嵌套               & 可以     & 可以     \\
    实现接口           & 可以     & 可以     \\
    产生和处理事件     & 可以     & 可以     \\
    \hline
  \end{tabular}
\end{figure}
\begin{enumerate}
\item 结构体不能定义无参数的构造函数
\end{enumerate}

\end{frame}

\section{基类的访问}

\begin{frame}[fragile]
\frametitle{基类与派生类的关系}
\begin{itemize}
\item 基类中的成员,除 sealed 成员,全部被派生类继承
\item 有 private 修饰的成员不可以被派生类访问
\vskip.5cm
\item 派生类的字段可以直接{\redwarn 隐藏}基类同名字段,或使用 new 关键字
\end{itemize}
\begin{lstlisting}[escapeinside=<>]
class Circle {
  public int R;
}
class Ring:Circle {
  public int R; // hide R in Circle
  public int r;
}
\end{lstlisting}
\end{frame}

\begin{frame}[fragile]
\frametitle{隐藏基类方法}
\begin{itemize}
\item 派生类的方法也可以直接隐藏基类方法,但会有编译警告
\item 要显式隐藏基类方法,派生类要用 new 关键字指出
\end{itemize}
  \lstset{emph={new}}
\begin{lstlisting}
class Circle      { public int R;
  public static double pi = 3.1415;
  public double Area()
    { return pi * R * R;}
}
class Ring:Circle { public int r;
  public new double Area()
    { return pi * R * R - pi * r * r; }
}
\end{lstlisting}

\end{frame}

\begin{frame}
\frametitle{多态性(\textit{Polymorphism})}

\begin{block}{多态性(\textit{Polymorphism})}
  \CJKindent 多态性就是多种表现形式,具体来说,可以用"一个对外接口,多个内在实现方法"表示。
\end{block}

  \begin{itemize}
    \setlength{\itemsep}{8pt plus 1pt}
  \item 同一个对象中,同一个名字的不同成员
  \item 同一个类中,方法的重载
  \item 子类与父类之间
    \begin{itemize}
    \item 同名字段的隐藏
    \item 同名方法的覆盖与隐藏
    \end{itemize}
  \item 运行时,使用同一个名字,动态访问不同的成员
  \end{itemize}
\end{frame}

\begin{frame}[fragile]
\frametitle{多态举例(1)}
\begin{itemize}
\item 字段的隐藏
\end{itemize}
\begin{columns}
\column{.4\textwidth}
\begin{lstlisting}
class A {
  public int x = 3;
}
\end{lstlisting}
\column{.4\textwidth}
\begin{lstlisting}
class B : A {
  public int x = 5;
}
\end{lstlisting}
\end{columns}
\begin{lstlisting}[escapeinside=<>]
   static void Main(string[] args)
   {
      B b = new B();
      A a = b;
      System.Console.WriteLine(b.x);  // <输出是?>
      System.Console.WriteLine(a.x);  // <输出是?>
   }
\end{lstlisting}
\end{frame}

\begin{frame}[fragile]
\frametitle{多态举例(2)}
\begin{itemize}
\item 方法的隐藏
\end{itemize}
\begin{columns}
\column{.4\textwidth}
\begin{lstlisting}
class A {
  public void func()
  { WriteLine("A"); }
}
\end{lstlisting}
\column{.4\textwidth}
\begin{lstlisting}
class B : A {
  public void func()
  { WriteLine("B"); }
}
\end{lstlisting}
\end{columns}
\begin{lstlisting}[escapeinside=<>]
   static void Main(string[] args)
   {
      B b = new B();
      A a = b;
      b.func();   // <输出为?>
      a.func();   // <输出为?>
   }
\end{lstlisting}
\end{frame}


\begin{frame}[fragile]
\frametitle{虚方法 (\textit{virtual method})}
\begin{itemize}
\item 如果允许派生类{\redwarn 覆盖}基类方法,要用 virtual 关键字指出
\item 同时派生类要用 override 指出
\item 虚方法,也可以使用 new 关键字隐藏
\end{itemize}
  \lstset{emph={virtual,override,Area}}
\begin{lstlisting}
class Circle      { public int R;
  public static double pi = 3.1415;
  public virtual double Area()
    { return pi * R * R; }
}
class Ring:Circle { public int r;
  public override double Area()
    { return pi * R * R - pi * r * r; }
}
\end{lstlisting}
\end{frame}

\begin{frame}[fragile]
\frametitle{隐藏和覆盖的区别}
\begin{itemize}
\item 隐藏方法,在基类中还存在;覆盖的,已经被替换
\end{itemize}
隐藏基类的方法:
\lstset{emph={new}}
\begin{lstlisting}[escapeinside=<>]
using System;
class A
{ public void Hello()
  { Console.WriteLine ("in class A"); }
}
class B:A
{ public new void Hello() // new <用于隐藏>
  { Console.WriteLine ("in class B"); }
} ...
  public static void Main()
  { B b = new B(); // <基类引用>
    A a = (A) b;   // <派生类引用>
    b.Hello();     // "in class B"
    a.Hello();     // "in class A"
    Console.ReadLine();
  }

\end{lstlisting}
\end{frame}

\begin{frame}[fragile]
\frametitle{隐藏和覆盖的区别(2)}
覆盖基类的方法:
\lstset{emph={virtual,override}}
\begin{lstlisting}[escapeinside=<>]
using System;
class A
{ public virtual void Hello()
  { Console.WriteLine ("in class A"); }
}
class B:A
{ public override void Hello() // override <用于覆盖>
  { Console.WriteLine ("in class B"); }
} ...
  public static void Main()
  {
    B b = new B(); // <基类引用>
    A a = (A) b;   // <派生类引用>
    b.Hello(); // "in class B"
    a.Hello(); // "in class B" !!
    Console.ReadLine();
  }

\end{lstlisting}
\end{frame}

\begin{frame}[fragile]
\frametitle{隐藏和覆盖的区别(3)}
\begin{columns}
  \column{.5\textwidth}
\begin{itemize}
\item virtual,允许派生类覆盖的虚方法
\item override,只能用于覆盖虚方法
\item new 可以隐藏基类方法
\vskip.5cm
\item 通过类型转换,可以使用基类型表示对象
\item 通过基类引用的对象,可以直接访问基类方法
\end{itemize}
  \column{.5\textwidth}
  \begin{figure}
    \centering
    %% Slides for ".NET Programming" by Chunyu Wang <chunyu@hit.edu.cn>
%% $Rev$ $LastChangedDate$

\begin{tikzpicture}[font=\footnotesize, >=stealth]
\tikzstyle{cn}=[circle,fill=red!20,draw,minimum size=1cm]
\draw[fill=blue!20] (0,2.5) rectangle (1.75,4.75);
\draw (0,3.75) rectangle (1.5,4.75);
\draw (.875,3) node {������~B} (.75,4.25) node {����~A};
\draw[font=\scriptsize,dashed] (4,4.25) node[cn,minimum size=1.1cm] (b1h) {Hello};
\draw[->,dashed] (1.5,4.25) -- node[pos=.6,above] {virtual} (b1h);
\draw[font=\scriptsize] (4,4.25) node[cn] (b2h) {Hello};
\draw[->] (1.75,3) -| node[pos=.3,above] {override} (b2h);

\draw[fill=blue!20] (0,0) rectangle (1.75,2.25);
\draw (0,1.25) rectangle (1.5,2.25);
\draw (.875,.5) node {������~B} (.75,1.75) node {����~A};
\draw[font=\scriptsize] (4,1.75) node[cn] (a1h) {Hello};
\draw[->] (1.5,1.75) -- (a1h);
\draw[font=\scriptsize] (4,.5) node[cn] (a2h) {Hello};
\draw[->] (1.75,0.5) -- node[pos=.6,above] {new}(a2h);
\end{tikzpicture}

  \end{figure}
\end{columns}
\end{frame}

\begin{frame}[fragile]
\frametitle{\texttt{this}关键字访问类成员}
\begin{itemize}
\item 使用this关键字表示当前类自身
\end{itemize}
\lstset{emph={this}}
\begin{lstlisting}
using System;
class Person {
  string name;
  public Person (string name) {
    this.name = name;
  }
  public void Introduce(Person a) {
    if (a!=this)
      Console.WriteLine("Hello, I'm "+name);
  }
}...
  public static void Main()
  {
    Person p = new Person("Zhang");
    Person q = new Person("Wang");
    p.Introduce (q);
  }
\end{lstlisting}
\end{frame}

\begin{frame}[fragile]
\frametitle{\texttt{base}关键字访问基类成员}
\begin{itemize}
\item 使用base关键字表示基类
\end{itemize}
\lstset{emph={base}}
\begin{lstlisting}
class Student:Person
{
  public Student(string name):base(name)
  {}
  public new void Introduce (Person a)
  {   base.Introduce (a);
      Console.WriteLine ("I'm from HIT!");
  }
}
...
  public static void Main()
  {
    Person  p = new Person("Zhang");
    Student q = new Student("Wang");
    q.Introduce (p);
    Console.ReadLine();
  }
\end{lstlisting}
\end{frame}

\section{命名空间}

\begin{frame}[fragile]
\frametitle{命名空间}
\begin{block}{命名空间(namespace)}
  \CJKindent 类、接口、委托等其他类型的一个逻辑上的组合,用来防止名字之间的命名冲突。
\end{block}
\pause
定义命名空间的一般格式
\begin{lstlisting}
namespace <name>{
...
}
\end{lstlisting}
\begin{itemize}
\item 命名空间用于组织代码,防止名字冲突
\item 命名空间可以嵌套
\item 命名空间可以分开定义
\item 如果没有指定命名空间,默认在 \texttt{global} 命名空间中
\end{itemize}
\end{frame}

\begin{frame}[fragile]
\frametitle{命名空间的声明}
\begin{lstlisting}
namespace Counter {
  // A simple countdown counter.
  class CountDown {
    int val;

    public CountDown(int n) {
      val = n;
   }

    public void reset(int n) {
      val = n;
    }
    public int count() {
      if(val > 0) return val--;
      else return 0;
    }
  }
}
\end{lstlisting}
\end{frame}

\begin{frame}[fragile]
\frametitle{命名空间的使用}
在命名空间外,需要指出全名,使用 ``.'' 连接
\begin{lstlisting}
class NSDemo {
  public static void Main() {
    // Notice how CountDown is qualified by Counter.
    Counter.CountDown cd1 = new Counter.CountDown(10);
    int i;
    ...
}
\end{lstlisting}
\end{frame}

\begin{frame}[fragile]
\frametitle{命名空间的嵌套}
\begin{lstlisting}[escapeinside=<>]
namespace Animals{
  namespace Birds{
    public class Sparrow{
      // <类的具体实现>
    }
  }
}
// <名字空间外的使用>
Animals.Birds.Sparrow s = new Animals.Birds.Sparrow();
\end{lstlisting}
\end{frame}

\begin{frame}[fragile]
\frametitle{\texttt{using} 导入命名空间}

\begin{itemize}
\item 使用命名空间(using)
  \lstset{emph={using}}
\begin{lstlisting}
using Animals.Birds;
...
  Sparrow s = new Sparrow();
\end{lstlisting}
\item 使用别名(alias),C\# 2.0 可以使用 ``::'' 表示命名空间的别名
  \lstset{emph={using}}
\begin{lstlisting}
using b = Animals.Birds;
...
  b.Sparrow s = new b.Sparrow();
  b::Sparrow s = new b::Sparrow();
\end{lstlisting}
\end{itemize}
\end{frame}

\begin{frame}[fragile]
\frametitle{分开定义}
\begin{itemize}
\item 同一个命名空间可以分布在不同文件中
\item 也可以是同一个文件的不同部分
\end{itemize}
\begin{lstlisting}
namespace Counter{
  class a { ... }
  class b { ... }
}
...
namespace Counter{
  class c { ... }
  class d { ... }
}
// Counter have 4 class now
\end{lstlisting}
\end{frame}

\begin{frame}[fragile]
\frametitle{符号、变量、类型、实例、对象}

\begin{lstlisting}
int i = 10;
Circle c = new Circle();
Circle b = null;
\end{lstlisting}

\begin{itemize}
\setlength{\itemsep}{8pt plus 1pt}
\item 符号 --- 标识符,编译器输入的源程序中
\item 变量 --- 进程中表示一个具体的量的名字
\item 类型 --- 数据结构的定义
\item 实例 --- 为数据结构分配的内存
\item 对象 --- ...
\end{itemize}

\end{frame}


% Local Variables:
% mode: LaTeX
% TeX-master: "part-02.tex"
% TeX-header-end: "% End-of-Header$"
% TeX-trailer-start: "% Start-of-Trailer$"
% coding: utf-8
% End:
